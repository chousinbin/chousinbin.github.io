%%%%%%%%%%%%%%%%%%%%%%%%%%%%%%%%%%%%%%%%%
% Medium Length Professional CV
% LaTeX Template
% Version 2.0 (8/5/13)
%
% This template has been downloaded from:
% http://www.LaTeXTemplates.com
%
% Original author:
% Trey Hunner (http://www.treyhunner.com/)
%
% Important note:
% This template requires the resume.cls file to be in the same directory as the
% .tex file. The resume.cls file provides the resume style used for structuring the
% document.
%
%%%%%%%%%%%%%%%%%%%%%%%%%%%%%%%%%%%%%%%%%

%----------------------------------------------------------------------------------------
%	PACKAGES AND OTHER DOCUMENT CONFIGURATIONS
%----------------------------------------------------------------------------------------

\documentclass{resume} % Use the custom resume.cls style
\usepackage[left=0.75in,top=0.6in,right=0.75in,bottom=0.6in]{geometry} % Document margins
\usepackage{xeCJK}  
\usepackage{color}
\usepackage[citecolor=green]{hyperref}
\usepackage{graphicx} % 用于插入图片

\name{Sinbin Zhou} % Your name
\address{chousinbin@gmail.com}
\begin{document}
%----------------------------------------------------------------------------------------
%	EDUCATION SECTION
%----------------------------------------------------------------------------------------

\begin{rSection}{教育背景}
%\textbf{大连交通大学}\hfill{软件工程专业(本科)}\hfill{2021.09 - 2025.07}
\begin{rSubsection}{大连交通大学}{\em2021.09 - 2025.07}{软件工程专业}{辽宁大连}
\item 平均学分绩: 92.00 \ \ 必修课平均学分绩: 92.42 \ \ 排名: 2/129 (前 2\%)
\item 英语水平:2023年6月CET4 (449)\ \ \ \ 2023年12月CET6 (426)\ \ \ \ 均为首次
\item 主修课程:数据结构(99) 数据库原理与应用(98) 软件工程(96) 计算机网络(97) 高等数学(98) 线性代数(93) 
\end{rSubsection}
\end{rSection}

%----------------------------------------------------------------------------------------
%	HONORS / AWARDS SECTION
%----------------------------------------------------------------------------------------

\begin{rSection}{荣誉奖项}
\begin{tabular}{ @{} >{\bfseries}l @{\hspace{6ex}} l }
第十三届蓝桥杯算法竞赛C/C++赛道省赛一等奖 & {\em 2022.04} \\
第十五届蓝桥杯算法竞赛C/C++赛道省赛一等奖 & {\em 2024.04} \\
第三届辽宁省大学生程序设计竞赛铜奖 & {\em 2022.12} \\
校一等优秀学生综合奖学金(排名:1/129) & {\em 2023.12} \\
校三好学生 & {\em 2023.12} \\
校级创赛奖项若干(主持)\\
\end{tabular}
\end{rSection}

%----------------------------------------------------------------------------------------
%	PROJECTS / RESEARCH EXPERIENCE SECTION
%----------------------------------------------------------------------------------------

\begin{rSection}{项目经历}

\begin{rSubsection}{进程调度和页面置换算法的实现}{\em 2023.05 - 2023.06}
{用 C++ 分别模拟实现了几种常见的进程调度算法和页面置换算法,总共约 800 LOC\\}
{代码地址: \rm \url{https://github.com/chousinbin/OS-Lab}}
\item[]
\begin{itemize}
\setlength\itemsep{-0.5em}
\item[-] 进程调度: 先来先服务, 最短剩余优先, 时间片轮转, 优先级
\item[-] 页面置换:先入先出, 最近最少使用, 二次交换
\end{itemize}
\end{rSubsection}

\begin{rSubsection}{个人博客的搭建与部署}{\em 2022.07 - 2022.08}
{使用Hugo博客框架和开源主题,在GitPages上成功搭建了个人博客网站,并配备了独立域名\\}
{博客地址: \rm \url{https://xinb.in}}
\item[]
\begin{itemize}
\setlength\itemsep{-0.5em}
\item[-] 配备了独立域名,计算机网络上的有关知识得到了实践,比如CDN、DNS、IP等知识
\item[-] 通过在Linux环境下调试、管理项目,对Linux命令以及Git相关操作得到了锻炼
\item[-] 通过阅读官方文档,成功在Linux上调试,并且远程部署,能够正常发布博客文章
\end{itemize}
\end{rSubsection}

\begin{rSubsection}{个人电子笔记仓库的维护}{\em 2022.09 - 今}
{大学以来坚持使用 MarkDown 书写电子学习笔记,涉及开发技术、算法竞赛、本科课程等方面\\}
{仓库地址: \rm \url{https://github.com/chousinbin/Notes}}
\item[]
\begin{itemize}
\setlength\itemsep{-0.5em}
\item[-] 用 \LaTeX{} 书写数学公式
\item[-] 用 Git 对笔记进行版本控制,笔记托管在 GitHub 上,该仓库 Commit 次数已达 160 余次
\end{itemize}
\end{rSubsection}

\end{rSection}

%----------------------------------------------------------------------------------------
%	SKILLS SECTION
%----------------------------------------------------------------------------------------

\begin{rSection}{个人技能}
\begin{rSubsection}
{编程能力}{}{}{}
\item[]
\begin{itemize}
\setlength\itemsep{-0.5em}
\item[-] 了解Linux常用命令,有 Linux 环境下对项目进行开发与调试经历
\item[-] 熟悉 C/C++ 和 Java 语言的基本语法,独自开发过2000 LOC 的图形界面程序
\item[-] 熟悉 Git/GitHub,能对项目进行版本控制, 包括创建远程仓库、代码提交、分支管理、版本回退等操作
\end{itemize}
\end{rSubsection}

\begin{rSubsection}
{文档能力}{}{}{}
\item[]
\begin{itemize}
\setlength\itemsep{-0.5em}
\item[-] 熟悉MarkDown文档写作,了解\LaTeX{}排版
\end{itemize}
\end{rSubsection}

\begin{rSubsection}
{兴趣爱好}{}{}{}
\item[]
\begin{itemize}
\setlength\itemsep{-0.5em}
\item[-] 编程、长跑、徒步、垂钓
\end{itemize}
\end{rSubsection}

\end{rSection}

\end{document}
